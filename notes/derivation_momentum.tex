\documentclass[9pt,fleqn,twoside]{article}
 
%---Packages---
\usepackage{a4wide}
\usepackage{graphics}
\usepackage{amsmath}
\usepackage{amsfonts}
\usepackage{amssymb}
\usepackage{amsthm}
\usepackage{fancyhdr}
\usepackage{graphicx}
\usepackage{color}
\usepackage[lofdepth,lotdepth]{subfig}
\usepackage{hyperref}
%---New commands etc---
\theoremstyle{plain}
\newtheorem{thm}{Theorem}[section]
\newtheorem{cor}[thm]{Corollary}
\newtheorem{lem}[thm]{Lemma}
\newtheorem{prop}[thm]{Proposition}
\usepackage{setspace}

\theoremstyle{definition}
\newtheorem{defn}{Definition}[section]
\newtheorem{alg}{Algorithm}
\theoremstyle{remark}
\newtheorem*{rem}{Remark}
\pagestyle{fancy}
\fancyfoot[C]{\thepage\ of 6}
\lhead{\it{Owen Evans}}
%---Settings---
\setlength{\parskip}{0.3em}
\renewcommand{\headrulewidth}{0pt}
%---the document---
\begin{document}
%\doublespacing
\pagenumbering{arabic} \\
\noindent
\subsection{Momentum Equation}
Conservation of mass is stated in the usual way. For low Reynolds
number conservation of
momentum becomes a force balance statement for respective phases:
\begin{equation} 
\begin{split}
\nabla \cdot [\phi \mathbf{\sigma_f}] + \phi \rho_f \mathbf{g} +
\mathbf{F_f} &= \mathbf{0}  \\
 \nabla \cdot [(1 - \phi) \mathbf{\sigma_s}] + (1 - \phi) \rho_s \mathbf{g} +
\mathbf{F_s} &= \mathbf{0}
\end{split}
\end{equation}
It is assumed that the interphase force between phases respects
Newton's third law of motion, i.e. $\mathbf{F_f}$ =
$-\mathbf{F_s}$. We assume the following form for $\mathbf{F_s}$:
\begin{equation}
\mathbf{F_s} = d(\mathbf{v} - \mathbf{V}) - P \nabla \phi +
\mathbf{F_{\sigma}}, 
\end{equation}
where $\mathbf{F_{\sigma}}$ is the force that arises from interfacial
surface tension. For the time being we make no additional assumptions
about the form of this term. The second unknown in the above is the
interfacial pressure, $P$. This is conventionally taken to be equal to
the mechanical pressure in the fluid, but instead we will take the
general form $P = P_f + P_b$. The fluid momentum equation becomes
(after taking $\mathbf{\sigma_f} = -P_f\mathbb{I}$)
\begin{equation}
\begin{split}
&-\nabla \cdot [\phi P_f\mathbb{I}] + \phi \rho_f \mathbf{g} - d(\mathbf{v} - \mathbf{V}) + P \nabla \phi -
\mathbf{F_{\sigma}} \\
=&-\nabla \cdot [\phi P_f\mathbb{I}] + \phi \rho_f \mathbf{g} -
d(\mathbf{v} - \mathbf{V}) + (P_f + P_b) \nabla \phi -
\mathbf{F_{\sigma}} = 0,
\end{split}
\end{equation}
which, after some rearrangement becomes
\begin{equation}
\phi(\mathbf{v} - \mathbf{V}) = -\frac{\phi^2}{d} \left [ \nabla P_f -
  \rho_f \mathbf{g} - \frac{P_b}{\phi} \nabla \phi +
  \frac{\mathbf{F_{\sigma}}}{\phi} \right ],
\end{equation}
which is similar the standard Darcay equation but with some extra terms.
Similarly, the solid momentum equation becomes (after taking
$\mathbf{\sigma_s} = \mathbf{\tau} - P_s \mathbb{I}$)
\begin{equation}
\nabla \cdot [(1 - \phi) \mathbf{\tau_s}] - \nabla \cdot [(1-\phi)
P_s\mathbb{I}] + (1 - \phi) \rho_s \mathbf{g} + d(\mathbf{v} -
\mathbf{V}) - (P_f + P_b) \nabla \phi +
\mathbf{F_{\sigma}} = 0.
\end{equation}
To obtain the desired solid momentum equationwe add the equations in
(1), by which the interphase force terms drop out entirely, and we are
left with
\begin{equation}
\nabla \cdot [(1 - \phi) \mathbf{\tau_s}] + \nabla [ \phi(P_s - P_f)]
- \nabla P_s + \bar{\rho}\mathbf{g} = 0,
\end{equation}
where $\bar{\rho}$ is the phase-avergaed density. We now introduce
another closure:
\begin{equation}
P_f - P_s = \zeta \nabla \cdot \mathbf{V} + P_a,
\end{equation}
so that
\begin{equation}
\nabla \cdot [(1 - \phi) \mathbf{\tau_s}] - \nabla [ \phi(\zeta \nabla \cdot \mathbf{V} + P_a)]
- \nabla P_s + \bar{\rho}\mathbf{g} = 0,
\end{equation}
which is no different in content to the standard momentum equation
(all of our other modifications have dropped out), except that there
is an extra $-\nabla (\phi P_a)$ terms floating about.
\subsection{Surface Energy}
We assume that $\mathbf{F}_{\sigma} =
\nabla E_c$ (NOTE: this might be $\mathbf{F}_{\sigma} = -
\nabla E_c$), where $E_c$ is the interfacial crack energy per unit
volume. Following Riley and Kohstedt (1991) we write
\begin{equation}
E_c = \frac{1}{V} (A_{ss} \gamma_{ss} + A_{sl} \gamma_{sl})
\end{equation}
where $\gamma_{ss}$ and $\gamma_{sl}$ are the respective solid--solid
and solid--liquid interfacial free energies. $V$ is the volume of a
representative cube of length $d$, so that $V=d^3$. $A_{ss}$ is the
average solid--solid surface area within the volume -- the average of
surface area of dry cracks times the total number of dry cracks. $A_{sl}$ is the average surface area of
fluid filled cracks within the volume times the number of fluid filled
cracks. We assume that wet and dry
cracks have the same dimensions, and that the average surface area of
a crack  is $c^2$. We let $N_t = N_d + N_w$ be the average number of cracks within
a volume, such that $N_d$ and $N_w$ are the respective average numbers
of dry and wet cracks. Then
\begin{align}
A_{ss} &= (N_t - N_w) \times c^2 \\
&= N_t c^2 \left( 1 - \frac{N_w}{N_t} \right) \\
&= N_t c^2 ( 1 - f_w),
\end{align}
where $f_w$ is the ratio of wet cracks to the total number of
cracks. Similarly
\begin{equation}
A_{sl} = N_t c^2 f_w . 
\end{equation}
So that
\begin{equation}
E_c = \frac{c^2 N_t}{d^3} \left ( (1 - f_w) \gamma_{ss} + f_w
  \gamma_{sl} \right) .
\end{equation}
We let $\alpha = \frac{c^2 N_t}{d^3}$, the specific surface area of
cracks within the volume.\\
\\
Now we want to relate the defined quantities to the porosity. if
$\delta$ is a characteristic crack length, then it makes sense to say
that
\begin{equation}
\phi = \frac{N_w c^2 \delta}{d^3} = f_w \alpha \delta .
\end{equation}
Assumption: all fluid exists in cracks, and nowhere else -- is this
right? Writing $E_c$ with porosity terms gives
\begin{equation}
E_c = \frac{1}{\delta} \left( (\alpha \delta - \phi) \gamma_{ss} +
  \phi \gamma_{sl} \right).
\end{equation}
Then 
\begin{equation}
\mathbf{F}_{\sigma} = \nabla E_c =  \frac{\gamma_{sl} -
  \gamma_{ss}}{\delta} \nabla \phi
\end{equation}
\subsection{Momentum ala Mckenzie: 1. Viscous case}
We rewrite (8) to isolate $P_s$ (which we will eliminate shortly):
\begin{equation}
\nabla \cdot [(1 - \phi) \mathbf{\tau_s}] - \nabla [ (1 - \phi) P_s \mathbb{I}] - \nabla [\phi P_f \mathbb{I}] +  \bar{\rho}\mathbf{g} = 0
\end{equation}
Using (7) this can becomes
\begin{equation}
\nabla \cdot [(1 - \phi) \mathbf{\tau_s}] - \nabla [ (1 - \phi)(P_f - P_a - \zeta \nabla \cdot \mathbf{V}) \mathbb{I}] - \nabla [\phi P_f \mathbb{I}] +  \bar{\rho}\mathbf{g} = 0
\end{equation}
The deviatoric stress tensor for the matrix phase is assumed to be that of an isotropic viscous fluid:
\begin{equation}
\mathbf{\tau_s} = \mu_s \left( \nabla \mathbf{V} + \nabla \mathbf{V}^{T} - \frac{2}{3} (\nabla \cdot \mathbf{V}) \mathbb{I} \right)
\end{equation}
Substituting this relation into the solid momentum equation above gives
\begin{equation}
\nabla \cdot [ (1 - \phi) \mu_s (\nabla \mathbf{V} + \nabla\mathbf{V}^{T} )] - \nabla \cdot [( 1- \phi)(P_f - P_a) \mathbb{I} ] + \nabla \cdot \left( (1 - \phi)(\zeta - \frac{2}{3} \mu_s ) \nabla \cdot \mathbf{V} \mathbb{I} \right) - \nabla ( \phi P_f \mathbb{I} ) + \bar{\rho} \mathbf{g} = 0
\end{equation}
This can be simplified by defining $\eta = (1 - \phi) \mu_s$ and $\xi = (1 - \phi)(\zeta - \frac{2}{3} \mu_s)$:
\begin{equation}
  \nabla \cdot [ \eta (\nabla \mathbf{V} + \nabla\mathbf{V}^{T} )]  + \nabla \left( \xi  \nabla \cdot \mathbf{V}  \right)  - \nabla P_f + \nabla[(1 - \phi)P_a] + \bar{\rho} \mathbf{g} = 0
\end{equation}
Going back to the liquid momentum equation (i.e. Darcy), we let $P_b=0$ and define $d = \frac{\phi^2 \mu_f}{K}$, where $K$ is the permeability. Also take $\mu = \mu_f$. Then the respective solid and liquid momentum equation are
\begin{equation}
\boxed{\phi(\mathbf{v} - \mathbf{V}) = -\frac{K}{\mu} \left [ \nabla P_f - \rho_f \mathbf{g} + \frac{\mathbf{F_{\sigma}}}{\phi} \right ]} 
\end{equation}
\begin{equation}
\boxed{\nabla \cdot [ \eta (\nabla \mathbf{V} + \nabla\mathbf{V}^{T} )]  + \nabla \left( \xi  \nabla \cdot \mathbf{V}  \right)  - \nabla P_f + \nabla[(1 - \phi)P_a] + \bar{\rho} \mathbf{g} = 0}
\end{equation}
 \subsection{2. Elastic solid rheology}
We alter the solid rheology  by expressing the stress-strain relationship as
\begin{equation}
\mathbf{\tau_s} = G \left( \nabla \mathbf{U} + \nabla \mathbf{U}^{T} - \frac{2}{3} (\nabla \cdot \mathbf{U}) \mathbb{I} \right),
\end{equation}
where $G$ is the elastic shear modulus. Equation (7) can be modified such that
\begin{equation}
P_f - P_s = K \nabla \cdot \mathbf{U} + P_a,
\end{equation} 
where $K$ is the elastic bulk modulus. This only changes the solid momentum equation symbolically, where $\eta$ and $\zeta$ are replaced by the elastic moduli:
\begin{equation}
\nabla \cdot [ G (\nabla \mathbf{U} + \nabla\mathbf{U}^{T} )]  + \nabla \left( \lambda  \nabla \cdot \mathbf{U}  \right)  - \nabla P_f + \nabla[(1 - \phi)P_a] + \bar{\rho} \mathbf{g} = 0
\end{equation}
where $\lambda = (1 - \phi)(K - \frac{2}{3} \mu)$ is the modified second lame parameter.
\subsection{Pressure-split}
Split the fluid pressure: $P_f = \mathcal{P} + P^{*} + P_l$, where $\mathcal{P} = \lambda \nabla \cdot \mathbf{U}$ is the compaction pressure, $P_l= -\rho_m g z$ (is this right?) is the lithospheric pressure and $P^{*}$ is the remaining contribution. Then taking the material derivative of $\mathcal{P}$:
\begin{equation}
\boxed{\frac{D}{Dt} \left (\frac{\mathcal{P}}{\lambda} \right) = \nabla \cdot \mathbf{V}} .
\end{equation}
Aside: going back to conservation of momentum. Assume no melting (or reaction) and apply the Boussinesq approximation to both phases. Adding the solid and liquid equations then says that
\begin{equation}
\nabla \cdot [ \phi \mathbf{v} + (1 - \phi) \mathbf{V} ] = 0 \implies \nabla \cdot \mathbf{V} = - \nabla \cdot [\phi(\mathbf{v} - \mathbf{V})] 
\end{equation}
Applying Darcy (23) eliminates $\mathbf{v}$:
\begin{equation}
\nabla \cdot \mathbf{V} = \frac{D}{Dt} \left (\frac{\mathcal{P}}{\lambda}\right) = \nabla \cdot \frac{K}{\mu} \left [ \nabla P_f - \rho_f \mathbf{g} + \frac{\mathbf{F_{\sigma}}}{\phi} \right ]
\end{equation}
which, after substituting the pressure-split becomes
\begin{equation}
\boxed{- \nabla \cdot \left( \frac{K}{\mu} \nabla \mathcal{P} \right ) + \frac{D}{Dt} \left (\frac{\mathcal{P}}{\lambda}\right) = \nabla \cdot\frac{K}{\mu} \left [ \nabla P_l + \nabla P^{*} - \Delta \rho \mathbf{g} + \frac{\mathbf{F_{\sigma}}}{\phi} \right ]}
\end{equation}
where $\Delta \rho = \rho_m - \rho_f$.
The solid mass conservation equation says that
\begin{align}
\frac{\partial \phi}{\partial t} + (\mathbf{V} \cdot \nabla \phi) &= (1-\phi) \nabla \cdot \mathbf{V} \\
\implies \frac{D \phi}{Dt} &= ( 1 - \phi) \cdot \mathbf{V}
\end{align}
which can be written in terms of the compaction pressure
\begin{equation}
\boxed{\frac{D \phi}{D t} = (1 - \phi) \frac{D}{Dt} \left (\frac{\mathcal{P}}{\lambda}\right)}
\end{equation}
The remaining pressure contribution is given by
\begin{equation}
\boxed{\nabla P^{*} = \nabla \cdot [ G (\nabla \mathbf{U} + \nabla\mathbf{U}^{T} )] + \nabla[(1 - \phi) P_a] + \phi \Delta \rho \mathbf{g}}
\end{equation}
\subsection{Scaling: P_a = E_c}
Equations (31) and (32) form a coupled system. We ignore gravitational
effects and by restricting to a 1D case, can also neglect terms
involving $\mathbf{U}$. We consider two cases: 1. $P_a = E_c$ and
2. $P_a = 0$. For case 1, equation (31) becomes (replacing $\lambda$
with $K_e$)
\begin{equation}
\frac{D}{Dt} \left (\frac{\mathcal{P}}{K_e}\right) = \nabla
  \cdot\frac{K}{\mu} \left [ \nabla \mathcal{P} + \nabla [ ( 1- \phi)
    E_c ] + \frac{\nabla E_c}{\phi} \right ]},
\end{equation}
and the porosity evolution equation is
\begin{equation}
\frac{D \phi}{D t} = (1 - \phi) \frac{D}{Dt} \left (\frac{\mathcal{P}}{K_e}\right)
\end{equation}
We make the following scalings:
\begin{equation}
K = K_0 \left( \frac{\phi^n}{\phi_0^n} \right), \: \: \mathcal{P} = K_e
\mathcal{P}^{'}, \: \: t = \frac{t^{'}}{T}, \: \: x = Lx^{'},
\end{equation}
where $K_0$ is the charactestic permeability and $\phi_0$ is the
charateristic porosity. Following this out we have (dropping primes)
\begin{equation}
\frac{D}{Dt} \left (\mathcal{P}\right) &= \nabla
  \cdot\frac{K_0 \phi^n}{\mu \phi_0^n T L^2} \left [ \nabla (
    K_e\mathcal{P}) + (1 - \phi) \nabla E_c - E_c \nabla \phi +
    \frac{\nabla E_c}{\phi} \right ] .\\
\end{equation}
The RHS can be written as
\begin{equation}
\nabla
  \cdot\frac{K_0 K_e \phi^n}{\mu \phi_0^n T L^2}  \nabla 
    \mathcal{P} + \nabla \cdot\frac{K_0 \phi^n}{\mu \phi_0^n T L^2} \left ( \frac{\gamma_{sl}
      - \gamma_{ss}}{\delta} \right)\left [ (1 - \phi) \nabla \phi - \phi \nabla \phi -
    \alpha \gamma_{ss} \left( \frac{\delta}{\gamma_{sl} - \gamma_{ss}}
    \right) \nabla \phi
    + \frac{\nabla \phi}{\phi} \right ]
\end{equation}
where (17) has been used. We introduce dimensionless constants:
\begin{equation}
A = \frac{K_0 K_e}{\mu \phi_0^n T L^2}, \: \: B =  \frac{K_0}{\mu
  \phi_0^n T L^2} \left ( \frac{\gamma_{sl}  - \gamma_{ss}}{\delta} \right), \: \: C =  \alpha \gamma_{ss} \left( \frac{\delta}{\gamma_{sl} - \gamma_{ss}}
    \right), 
\end{equation}
which gives (after a bit of rearrangement)
\begin{equation}
\frac{D}{Dt} \left (\mathcal{P}\right) &= A \nabla
  \cdot [\phi^n \nabla \mathcal{P}] + B \nabla \cdot \phi^n [ ( 1- C)
  \nabla \phi - 2 \phi  \nabla \phi + \frac{\nabla \phi}{\phi} ],  
\end{equation}
which can be written as
\begin{equation}
\boxed{\frac{D \mathcal{P} }{Dt} &= A \nabla
  \cdot [\phi^n \nabla \mathcal{P}] + B \nabla \cdot \phi^n \left[ ( 1- C)
  \nabla \phi - \nabla (\phi^2) + \frac{\nabla \phi}{\phi} \right]},
\end{equation}
where $\nabla = \frac{d}{dx}$. To close the system we also have
\begin{equation}
\boxed{\frac{D \phi}{D t} = (1 - \phi) \frac{D \mathcal{P}}{Dt}}
\end{equation}
\subsection{P_a = 0}
In the case where$P_a = 0$, the above system reduces to
\begin{equation}
\boxed{\frac{D \mathcal{P} }{Dt} &= A \nabla
  \cdot [\phi^n \nabla \mathcal{P}] + B \nabla \cdot \phi^n \left[\frac{\nabla \phi}{\phi} \right]},
\end{equation}
and equation (44) is unchanged.
\subsection{Stability Analysis}
Take $P_a=0$ and perturb the coupled system (??) about a base
state. Any pair of constants $\phi = \phi_0$ and $\mathcal{P} =
\mathcal{P}_0$ is a solution (i.e. the system involves only gradients
and time derivatives, so clearly and constant is a
solution). This system is then perturbed about the base state via a
small parameter $\epsilon$:
 \begin{align}
\phi &= \phi_0 + \epsilon \phi_1 \\
\mathcal{P} &= \mathcal{P}_0 + \epsilon \mathcal{P}_1
\end{align}
At $\mathcal{O}(\epsilon)$ equation (44) becomes
\begin{equation}
\frac{D \phi_1}{D t} = (1 - \phi_0) \frac{D \mathcal{P}_1}{Dt},
\end{equation}
and equation (45) similarly becomes (after dividing by A)
\begin{equation}
\frac{1}{A} \frac{D \mathcal{P}_1 }{Dt} &=  \nabla
  \cdot [\phi_0^n \nabla \mathcal{P}_1] + \frac{B}{A} \nabla \cdot
  \left[ \phi_0^{n-1} \nabla \phi_1 \right]
\end{equation}
Treating the material derivatives in (48) as pure time derivatives, we can write
\begin{equation}
\phi_1 = (1 - \phi_0) \mathcal{P}_1 + c_1,
\end{equation}
for some constant $c_1$. Substituting this into (49) gives
\begin{equation}
\frac{1}{A(1 - \phi_0)} \frac{D \phi_1 }{Dt} &=  \nabla
  \cdot [\frac{\phi_0^n}{1 - \phi_0} \nabla \phi_1] + \frac{B}{A} \nabla \cdot
  \left[ \phi_0^{n-1} \nabla \phi_1 \right]
\end{equation}
We search for modes of the form $\phi_1 = \tilde{\phi}e^{\omega t + i
  k x}$. Substituting this into (51) gives
\begin{equation}
\frac{\omega}{A(1 - \phi_0)} = -k^2 \left( \frac{\phi_0^n}{1 - \phi_0}
  + \frac{B}{A} \phi_0^{n-1} \right)
\end{equation}
so that
\begin{equation}
\omega = -k^2( A \phi_0^n + B(1 - \phi_0)\phi_0^{n-1}).
\end{equation}
The sign of $\omega$ depends on $A,B$ and $\phi_0$:
\begin{itemize}
\item{({\it unstable}) $\omega>0$ if $A \phi_0^n + B(1 - \phi_0)\phi_0^{n-1} <0$}
\item{({\it stable}) $\omega<0$ if $A \phi_0^n + B(1 - \phi_0)\phi_0^{n-1} >0$}
\item{({\it neutrally stable}) $\omega=0$ if $A \phi_0^n + B(1
    -\phi_0)\phi_0^{n-1} = 0$.}
\end{itemize}
\begin{figure}[!htbp]
\centering
\scalebox{0.5} 
{\includegraphics{stability.pdf}} 
\caption{\small {\it Stability map for $\frac{B}{A}$ vs. $\phi_0$.}} 
\end{figure}
More simply, the system is stable if $\frac{\phi_0}{1-\phi_0} +
\frac{B}{A} > 0$, and is unstable if $\frac{\phi_0}{1-\phi_0} +
\frac{B}{A} > 0$. The neutral curve is then defined by $\frac{B}{A} =
\frac{\phi_0}{\phi_0 - 1}$. Plotting $\phi_0$ on the x-axis and
$\frac{B}{A}$ on the y-axis we obtain stability map given in Figure
1. 




\end{document}
