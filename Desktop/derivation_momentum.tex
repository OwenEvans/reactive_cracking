\documentclass[9pt,fleqn,twoside]{article}
 
%---Packages---
\usepackage{a4wide}
\usepackage{graphics}
\usepackage{amsmath}
\usepackage{amsfonts}
\usepackage{amssymb}
\usepackage{amsthm}
\usepackage{fancyhdr}
\usepackage{graphicx}
\usepackage{color}
\usepackage[lofdepth,lotdepth]{subfig}
\usepackage{hyperref}
%---New commands etc---
\theoremstyle{plain}
\newtheorem{thm}{Theorem}[section]
\newtheorem{cor}[thm]{Corollary}
\newtheorem{lem}[thm]{Lemma}
\newtheorem{prop}[thm]{Proposition}
\usepackage{setspace}

\theoremstyle{definition}
\newtheorem{defn}{Definition}[section]
\newtheorem{alg}{Algorithm}
\theoremstyle{remark}
\newtheorem*{rem}{Remark}
\pagestyle{fancy}
\fancyfoot[C]{\thepage\ of 2}
\lhead{\it{Owen Evans}}
%---Settings---
\setlength{\parskip}{0.3em}
\renewcommand{\headrulewidth}{0pt}
%---the document---
\begin{document}
%\doublespacing
\pagenumbering{arabic} \\
\noindent
\subsection{Momentum Equation}
Conservation of mass is stated in the usual way. For low Reynolds
number conservation of
momentum becomes a force balance statement for respective phases:
\begin{equation} 
\begin{split}
\nabla \cdot [\phi \mathbf{\sigma_f}] + \phi \rho_f \mathbf{g} +
\mathbf{F_f} &= \mathbf{0}  \\
 \nabla \cdot [(1 - \phi) \mathbf{\sigma_s}] + (1 - \phi) \rho_s \mathbf{g} +
\mathbf{F_s} &= \mathbf{0}
\end{split}
\end{equation}
It is assumed that the interphase force between phases respects
Newton's third law of motion, i.e. $\mathbf{F_f}$ =
$-\mathbf{F_s}$. We assume the following form for $\mathbf{F_s}$:
\begin{equation}
\mathbf{F_s} = d(\mathbf{v} - \mathbf{V}) - P \nabla \phi +
\mathbf{F_{\sigma}}, 
\end{equation}
where $\mathbf{F_{\sigma}}$ is the force that arises from interfacial
surface tension. For the time being we make no additional assumptions
about the form of this term. The second unknown in the above is the
interfacial pressure, $P$. This is conventionally taken to be equal to
the mechanical pressure in the fluid, but instead we will take the
general form $P = P_f + P_b$. The fluid momentum equation becomes
(after taking $\mathbf{\sigma_f} = -P_f\mathbb{I}$)
\begin{equation}
\begin{split}
&-\nabla \cdot [\phi P_f\mathbb{I}] + \phi \rho_f \mathbf{g} - d(\mathbf{v} - \mathbf{V}) + P \nabla \phi -
\mathbf{F_{\sigma}} \\
=&-\nabla \cdot [\phi P_f\mathbb{I}] + \phi \rho_f \mathbf{g} -
d(\mathbf{v} - \mathbf{V}) + (P_f + P_b) \nabla \phi -
\mathbf{F_{\sigma}} = 0,
\end{split}
\end{equation}
which, after some rearrangement becomes
\begin{equation}
\phi(\mathbf{v} - \mathbf{V}) = -\frac{\phi^2}{d} \left [ \nabla P_f -
  \rho_f \mathbf{g} - \frac{P_b}{\phi} \nabla \phi +
  \frac{\mathbf{F_{\sigma}}}{\phi} \right ],
\end{equation}
which is similar the standard Darcay equation but with some extra terms.
Similarly, the solid momentum equation becomes (after taking
$\mathbf{\sigma_s} = \mathbf{\tau} - P_s \mathbb{I}$)
\begin{equation}
\nabla \cdot [(1 - \phi) \mathbf{\tau_s}] - \nabla \cdot [(1-\phi)
P_s\mathbb{I}] + (1 - \phi) \rho_s \mathbf{g} + d(\mathbf{v} -
\mathbf{V}) - (P_f + P_b) \nabla \phi +
\mathbf{F_{\sigma}} = 0.
\end{equation}
To obtain the desired solid momentum equationwe add the equations in
(1), by which the interphase force terms drop out entirely, and we are
left with
\begin{equation}
\nabla \cdot [(1 - \phi) \mathbf{\tau_s}] + \nabla [ \phi(P_s - P_f)]
- \nabla P_s + \bar{\rho}\mathbf{g} = 0,
\end{equation}
where $\bar{\rho}$ is the phase-avergaed density. We now introduce
another closure:
\begin{equation}
P_f - P_s = \zeta \nabla \cdot \mathbf{V} + P_a,
\end{equation}
so that
\begin{equation}
\nabla \cdot [(1 - \phi) \mathbf{\tau_s}] - \nabla [ \phi(\zeta \nabla \cdot \mathbf{V} + P_a)]
- \nabla P_s + \bar{\rho}\mathbf{g} = 0,
\end{equation}
which is no different in content to the standard momentum equation
(all of our other modifications have dropped out), except that there
is an extra $-\nabla (\phi P_a)$ terms floating about.
\subsection{Surface Energy}
We assume that $\mathbf{F}_{\sigma} =
\nabla E_c$, where $E_c$ is the interfacial crack energy per unit
volume. Following Riley and Kohstedt (1991) we write
\begin{equation}
E_c = \frac{1}{V} (A_{ss} \gamma_{ss} + A_{sl} \gamma_{sl})
\end{equation}
where $\gamma_{ss}$ and $\gamma_{sl}$ are the respective solid--solid
and solid--liquid interfacial free energies. $V$ is the volume of a
representative cube of length $d$, so that $V=d^3$. $A_{ss}$ is the
average solid--solid surface area within the volume -- the average of
surface area of dry cracks times the total number of dry cracks. $A_{sl}$ is the average surface area of
fluid filled cracks within the volume times the number of fluid filled
cracks. We assume that wet and dry
cracks have the same dimensions, and that the average surface area of
a crack  is $c^2$. We let $N_t = N_d + N_w$ be the average number of cracks within
a volume, such that $N_d$ and $N_w$ are the respective average numbers
of dry and wet cracks. Then
\begin{align}
A_{ss} &= (N_t - N_w) \times c^2 \\
&= N_t c^2 \left( 1 - \frac{N_w}{N_t} \right) \\
&= N_t c^2 ( 1 - f_w),
\end{align}
where $f_w$ is the ratio of wet cracks to the total number of
cracks. Similarly
\begin{equation}
A_{sl} = N_t c^2 f_w . 
\end{equation}
So that
\begin{equation}
E_c = \frac{c^2 N_t}{d^3} \left ( (1 - f_w) \gamma_{ss} + f_w
  \gamma_{sl} \right) .
\end{equation}
We let $\alpha = \frac{c^2 N_t}{d^3}$, the specific surface area of
cracks within the volume.\\
\\
Now we want to relate the defined quantities to the porosity. if
$\delta$ is a characteristic crack length, then it makes sense to say
that
\begin{equation}
\phi = \frac{N_w c^2 \delta}{d^3} = f_w \alpha \delta .
\end{equation}
Assumption: all fluid exists in cracks, and nowhere else -- is this
right? Writing $E_c$ with porosity terms gives
\begin{equation}
E_c = \frac{1}{\delta} \left( (\alpha \delta - \phi) \gamma_{ss} -
  \phi \gamma_{sl} \right).
\end{equation}
Then 
\begin{equation}
\mathbf{F}_{\sigma} = \nabla E_c = - \frac{\gamma_{ss} +
  \gamma_{sl}}{\delta} \nabla \phi
\end{equation}
\end{document}